%\lesson{6}{wo 30 okt 2019 16:10}{}
\begin{prop}\label{prop:usedfol}
    Let $V_1, \ldots, V_k$ linearly independent vector fields on $M$ such that $ [V_i, V_j] = 0$. Then for every  $p \in M$ there is a chart $(U, (s_{1}, \ldots, s_{n}))$ centered at $p$ such that $V_i = \frac{\partial }{\partial s_i} $ for $i = 1, \ldots, k$.
\end{prop}
\begin{proof}
    Since this is a local statement, let's assume that we're working in $\R^{n}$, and $p = 0$.
    Assume that $\spann_i \{V_i|_0\} \oplus (\{0\} \times \R^{n-k}) = \R^n$.
    Denote by $\theta_i$ the flow of $V_i$ and let $\Omega$ be a small neighborhood of the origin in $\{0\} \times \R^{n-k}$.
    Define  
    \begin{align*}
        \Phi: (-\epsilon, \epsilon)^{k} \times \Omega &\longrightarrow  \R^n\\
        (s_1, \ldots, s_k, s_{k+1}, \ldots, s_n)&\longmapsto
        (\theta_1)_{s_1}  \circ  \cdots  \circ 
        (\theta_k)_{s_k}
        (0, \ldots, 0, s_{k+1},\ldots, s_n)^{T}
    .\end{align*}
 %   Remember: the only thing we know is that $V_i$ is independent and they commute.
    Then $\frac{\partial }{\partial x_i} $ is $\Phi$-related to $V_i$ for $i \le  k$.
   Indeed for for $i \le  k$ you can compute that
$$(D_{(s_1,\dots,s_n)}\Phi)\Big(  \frac{\partial }{\partial s_i}\Big)=V_i|_{\Phi(s_1,\dots,s_n)}$$
   by using the curve $\epsilon\mapsto (s_1,\dots,s_i+\epsilon,\dots,s_n)$ and using the fact that the flows $\theta_i$ commute by the last corollary. At the point $p=0$ we clearly have
$(D_{0}\Phi)(  \frac{\partial }{\partial s_i}) =  \frac{\partial }{\partial x_i}|_0$
for $i>k$, hence $d_0 \Phi$ is an isomorphism.
    By the inverse function theorem, there exists a neighbourhood $V$ of $0$ such that $\Phi|_V: V \to  \Phi(V)=:U$ is a diffeomorphism. The desired chart is the inverse of this diffeomorphism.
\end{proof}
\begin{figure}[H]
    \centering
    \incfig{canonical-form-of-commuting-vector-fields}
    \caption{
        Two examples illustrating the proof.
        On the left, $n = k = 2$.
        On the right, $n = 3$ and  $k = 2$.
    }
    \label{fig:canonical-form-of-commuting-vector-fields}
\end{figure}



\setcounter{chapter}{3}
\chapter{Bundles}

\section{Fiber bundles}

\begin{definition}[Fiber bundle]
    Let $F$ be a manifold.
    A fiber bundle with fiber $F$ is a smooth surjective map $$\pi: E \to  B$$ between manifolds s.t.\ for all $x \in B$, there exists an open neighbourhood $U$ and a diffeomorphism $\psi: \pi^{-1}(U) \to  U \times F$ making this diagram commute:
    \[
        \begin{tikzcd}[ column sep=1em ]
          \pi^{-1}(U) \arrow[rr, "\psi"] \arrow[dr, "\pi"]& &U \times F \arrow[dl, "\pi_1"]\\
                                                     & U &
        \end{tikzcd}
    \]
\end{definition}

\begin{figure}[H]
    \centering
    \incfig{example-of-fiber-bundle}
    \caption{Example of a product fiber bundle}
    \label{fig:example-of-fiber-bundle}
\end{figure}



\begin{remark}
    The map $\pi: E \to  B$ is a submersion, i.e.\ $(\pi_*)_e$ is surjective in all points $e$ of $E$.
    Indeed: locally, the map $\pi$ looks like the projection onto the first factor $\pi$, as $\psi$ is a diffeomorphism.
    So we can work with $\pi_1$, which is a submersion.
\end{remark}
\begin{remark}
    For all $x \in B$, the \emph{fiber}  $\pi^{-1}(x)$ is diffeomorphic to $F$.
    Indeed, $\pi^{-1}(x) \approx \pi_1^{-1}(x) = \{x\} \times F \approx F$.
    So we have a family of manifolds (the fibers) which all are diffeomorphic to $F$, and this family is parametrized by $B$.

$E$ is called the \emph{total space}, $B$ the \emph{base space} and $\psi$ a \emph{local trivilization}.
A \emph{section} is a smooth  $f: B \to  E$ such that $\pi  \circ  f = Id_B$.
%Locally, it looks like the graph of a function.
\end{remark}

\begin{figure}[H]
    \centering
    \incfig{definition-of-a-section}
    \caption{Definition of a section}
    \label{fig:definition-of-a-section}
\end{figure}

\begin{eg}
    Given $F, B$ the product bundle is $B \times F \xrightarrow{\pi_1}   B$.
\end{eg}
\begin{eg}
    Let $F$ be a manifold, $\phi: F \to F$ diffeomorphism.
    Define the manifold $E := ([0, 1] \times F) / \sim$ and  $B = S^{1}$, where $\sim$ is given by
    \[
        (0, p) \sim (1, \phi(p))
    .\] 
Then $\pi:E\to B$ is a fiber bundle with typical fiber $F$.
    For instance, take $F = (-\epsilon, \epsilon)$, $\phi = -\Id$. Then $E$ is the M\"obius strip.
\begin{figure}[H]
    \centering
    \incfig{mobius-strip}
    \caption{M\"obius strip}
    \label{fig:mobius-strip}
\end{figure}
\end{eg}

\begin{eg}
The first projection    $\pi_1:\R^2\setminus\{0\}\to \R, (x,y)\mapsto x$ is \emph{not} a fiber bundle.
For instance, not all fibers are diffeomorphic.
\end{eg}

\filbreak
\section{Vector bundles}
\begin{definition}[Vector bundle]
    Fix $n \in  \N_{\ge 0}$. A vector bundle of rank $n$ is a smooth surjective map $\pi: E \to  B$ between manifolds $E, B$ such that 
    \begin{itemize}
        \item $E_p = \pi^{-1}(p)$ is a $n$-dimensional vector space for all $p \in B$ 
        \item For all $p \in B$, there exist a neighborhood $U$ of $p$ and a diffeomorphism $\psi: E|_U:=\pi^{-1}(U) \to  U \times \R^n$ such that the following diagram commutes
    \[
        \begin{tikzcd}[ column sep=1em ]
            \pi^{-1}(U)\arrow[rr, "\psi"] \arrow[dr, "\pi"]& &U \times \R^n \arrow[dl, "\pi_1"]\\
                                                     & U &
        \end{tikzcd}
    \]
    and $\psi|_{E_q}: E_q \to  \{q\} \times \R^n$ is a linear isomorphism, for all $q\in U$.
    \end{itemize}
\end{definition}

\begin{figure}[H]
    \centering
    \incfig{vector-bundle-definition}
    \caption{Definition of a vector bundle}
    \label{fig:vector-bundle-definition}
\end{figure}


\begin{remark}
 Equivalently, a  vector bundle of rank $n$ is a fiber bundle s.t.\ each fiber as an $n$-dimensional vector space and trivializations are linear in each fiber.
\end{remark}

\begin{remark}
    \begin{itemize}
        \item There exists a canonical section $B \to  E, q \mapsto (\text{zero vector in $E_q$})$. It is called zero section.
        \item Denote by $\Gamma(E)$ the set of all sections of $E$, it is a vector space.
            Note that we can also multiply a function $B \to  \R$ with a section $B \to  E$.
         Hence  $\Gamma(E)$ is a module over the algebra $C^{\infty}(B)$.
      %      Even more, it's a projective module!
    \end{itemize}
\end{remark}
\begin{eg}
    Given a manifold, take $B \times \R^n \xrightarrow{\pi_1} B$.
    Notice $\Gamma(B \times \R^n) =
    C^{\infty}(B, \R^n)$.
\end{eg}
\begin{prop}
    Let $B$ be a manifold of dimension $n$.
    Then 
    \[
    TB := \bigsqcup_{p \in B} T_p B
    \] 
    is naturally a vector bundle of rank $n$, called the \emph{tangent bundle of $B$}
\end{prop}
\begin{proof}
    Define $\pi: TB \to  B, v \in T_p B \mapsto p$.
    Notice that $\pi^{-1}(p) = T_pB$ is a vector space.
    
  We show that $TB$ is a manifold.
    Consider a chart $\phi: U \to  \phi(U)$ of $B$, denote its components by $(x_1,\dots,x_n)$.
    It induces a   bijection
    \begin{align*}
        \psi: (TB)|_U &\longrightarrow U\times \R^n \\
        a_{i} \frac{\partial }{\partial x_i}\Big|_p
                      &\longmapsto (p, a_{i})
    .\end{align*}
    Now take a cover of $B$ by charts $(U_\alpha, \phi_\alpha)_{\alpha \in A}$.
    Notice that $\psi_\beta  \circ  \psi_\alpha^{-1}$ is a smooth map from $(U_\alpha\cap U_\beta)\times \R^n$ to itself.
    Consider the topology on $TB$ with basis  $\psi_\alpha^{-1}(\sigma)$ for $\sigma \subset U_\alpha \times \R^n$ open and $\alpha \in A$.
    Then $\{((TB)|_{U_\alpha}), \psi_a\} $ is a\footnote{Strictly speaking, $\psi_{\alpha}$ is not a chart, because $U_\alpha \times \R^n$ is not an open subset of $\R^n\times \R^n$. But we can identify  $U_\alpha$ with $\phi_{\alpha}(U_\alpha)$, which is an open subset of $\R^n$.}    
     smooth atlas.
    Hence $TB$ is a smooth manifold.
    For all $\alpha \in A$, the chart $\psi_\alpha: (TB)|_{U_\alpha} \to  U_\alpha \times \R^n$ is a diffeomorphism and linear on every fiber.
\end{proof}
 

 
%\begin{remark}
%    Note that we've \emph{used} the charts to define a topology on $TB$!
%\end{remark}
%\begin{remark}
%    $\psi$ is a bijection, homeomorphism after adding the topology.
%    But also smooth structure!
%    And even more: vector bundle structure!
%\end{remark}

\begin{remark}
    $\Gamma(TM) = \mathfrak{X}(M)$
\end{remark}
