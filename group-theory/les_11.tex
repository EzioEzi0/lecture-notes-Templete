\lesson{11}{di 03 nov 2020 21:07}{}

First some facts which you can just assume to be true.

If $G$ is a finite group and $|G| = p^{k}m$,
with $\gcd(p, m) = 1$.
Then  $G$ contains a subgroup $P$ of order $p^{k}$.

If both $P, P' \le G$ and $|P| = |P'| = p^{k}$, then there exists a $g \in G$ such that $P = g P' g^{-1}$.

$P$ is a Sylov $p$-subgroup.

More facts:
\begin{lemma}
    If $G$ is a finite group and $P$ is a Sylov $p$-subgroup,
    then $N_G(P)$ is its own normalizer.
So $N_G(N_G(P)) = N_G(P)$.
\end{lemma}

 \begin{proof}
     Recall, $N_G(P) = \{g \in G  \mid  g P g^{-1} =P\} $ .
     Now, consider $g \in N_G(N_G(P))$.
     Then  $P \le  N_G(P)$ (this is always the case) and $P$ is a Sylov $p$-subgroup of $N_G(P)$. (Check this, easy)
     Now, $P' =  g P g^{-1} \le  g N_G(P) g^{-1} = N_G(P)$, because $g \in N_G(N_G(P))$.
     So $P' \le N_G(P)$ so $P'$ is also a Sylov $p$-subgroups of $N_G(P)$.
     So this means that these are conjugate inside  $N_G(P)$. So there exists an  $h  \in N_G(P)$ such that $P' = h P h^{-1}$.
     But as $h \in N_G(P)$, we have that $P' = P$.
     This shows that $P' = P$, so  $g P g^{-1} = P$, so $g \in N_G(P)$.
\end{proof}

We will use this fact later on.

Another trivial observation.
\begin{lemma}
    Let $N, M \triangleleft G$ and assume that $N \cap M = 1$.
    Then $[N, M] = 1$.
\end{lemma}
\begin{proof}
    $[N, M] \subset N$ because $N \triangleleft G$, and similarly for $[N, M] \subset M$.
    As the intersection is $1$, we get that $[N, M] = 1$.
\end{proof}

\begin{theorem}[17.1.4]
    Let $G$ be a finite group. Then TFAE
    \begin{enumerate}[(1)]
        \item $G$ is nilpotent
        \item Any subgroup $H \le G$ is subnormal
        \item  $G$ is the direct product of its Sylov $p$-subgroups
        \item (Something else in the text $\Phi(G)$ \ldots)
    \end{enumerate}
\end{theorem}
\marginpar{16:00}
\begin{proof}
    Remember we've proved already $(1) \implies(2)$.
    We also have proved $(3) \implies (1)$.
    If $G$ is a direct product of Sylov $p$-groups then we have product of nilpotents which is nilpotent.

    So let's prove that $(2) \implies (3)$.
    Let $P \le G$ be a Sylov $p$-subgroup.
    Then $N_G(N_G(P)) = N_G(P)$.
    So  $N_G(P)$ is only normal in itself and not in a larger subgroup of $G$.\footnote{Normalizer gives largest group that your group is normal in}

    So $N_G(P) \neq G$, then this would imply that that  $N_G(P)$ is not subnormal, which would then contract $(2)$.
    So $N_G(P)  = G$. So $P \triangleleft G$, i.e.\ any Sylov $p$-subgroup is normal in $G$.

    (Note: This also means that Sylov $p$-subgroups are unique. For any $p  \mid  |G|$ there is only one Sylov $p$-subgroup.)

    Let $ P_1, P_2, \ldots, P_r$ be the different Sylov $p$-subgroups of $G$.
    So
    \[
    |G| = p_1^{n_1} \cdots p_r^{n_r}
    ,\] 
    where $|P_i| = p_i^{n_i}$.
    Now, $P_i \cap P_j = 1$, because $P_i$ only contains elements of a certain power and  $P_j$ elements of another power.
    (And even more is true  $P_i \cap P_1 \cdots \hat{P_i} \cdots P_n = 1$, but not needed here)
    This implies that $[P_i, P_j] = 1$.
    Consider
    \[
    P_1 P_2 \cdots P_r \le G
    .\] 
    By commutativity of these elements we get that
    \[
    |P_1 P_2 \cdots P_r| = |G|
    ,\] 
    so this implies that $ P_1 P_2 \cdots P_n$,
    and as these are commutative we have that 
    \[
    G \cong P_1 \times P_2 \times \cdots  \times P_r
    .\] 
\end{proof}

\begin{remark}
    So for finite groups we have this.
    For infinite groups, $(2) \implies (1)$ is not true!
\end{remark}

\nsection{17}{2}{Finitely generated nilpotent groups}

Recall from last time,
\begin{align*}
    [ab, c] &= [a, c]^{b} [b, c]\\
            &= [a,c] [a,c]^{-1}b^{-1}[a,c][b,c]\\
            &= [a,c][[a,c], b] [b, c]
.\end{align*} 

Assume $a, b \in \gamma_j(G)$ and $c \in \gamma_i(G)$.
Then one of problems: $[\gamma_i(G), \gamma_j(G)] \le \gamma_{i+j}(G)$.
So then looking at $[ab, c]$ mod  $\gamma_{i+j+1}(G)$, we gave that
\[
    [a,c] \underbrace{[[a,c], b]}_{2i + j \text{layer}} [b, c]  = [a, c] [b, c] \mod \gamma_{i+j+1}(G)
.\] 

So
\[
    [ab, c] \gamma_{i+j+1}(G) = [a, c][b, c] \gamma_{i+j+1}(G)
.\] 
More in general, we can write
\[
    [ab, cd] = [a,c] [a,d] [b,c] [b, d] \mod \gamma_{i+j+1}(G)
\] 
if $a, b \in \gamma_i$ and $c, d \in \gamma_j$.
So $[\cdot , \cdot ]$ is bilinear upto $i+j+1$.
So also 
 \[
     [a^{-1}, c] = [a, c]^{-1} \mod \gamma_{i+j+1}(G)
.\] 


\begin{lemma}[17.2.1]
    Let $G$ be a group generated by a set $M$.
    Then $\gamma_i(G)$ is generated by all elements of the form
    \[
        [ \cdots [ m_1, m_2], m_3], \cdots, m_i]
    ,\] 
    where $m_{i} \in M$ together with $\gamma_{i+1}$.
\end{lemma}
\begin{proof}
    We do this by induction on $i$.
    When $i = 1$, $G$ is generated by all elements of $M$.
    Now assume okay for $\gamma_i(G)$.

    By induction, an element $\gamma_i(G)$ is of the form
    \[
    x = x_1^{\epsilon_1} \cdots x_n^{\epsilon_n} z
    ,\] 
    where each $x_j$ is an  $i$-th fold bracket, $\epsilon$ is  $\pm 1$ and  $z \in \gamma_{i+1}$. We can put this at the back because the group is normal.

    An element $y \in G$ is a product

    \[
    y = a_1^{\mu_1} \cdots a_s^{\mu_s}
    ,\] 
    where $a_j \in M$ and $\mu_j = \pm 1$.
    Now, $\gamma_{i+1} = [\gamma_i(G), G]$, by is by definition generated by all elements of the form $[x, y]$, where  $x \in \gamma_i$ and $y \in G$.
    Then
    \[
        [x, y] \gamma_{i+2}(G) = 
        [ x_1^{\epsilon_1} \cdots x_n^{\epsilon_n} z, a_1^{\mu_1} \cdots a_s^{\mu_s}] \gamma_{i+2}(G)
    .\] 
    Now the first argument is in $\gamma_i$ and the second in  $\gamma_1$, so mod  $\gamma_{i+1+1}$ this is bilinear, so we get
    \[
        [x, y] \gamma_{i+2}(G) = \prod [x_i, a_j]^{\epsilon_i \mu_j} \prod [z, a_j]^{\mu_j} \gamma_{i+2}(G)
    ,\] 
    but $[z, a_j]^{\mu_j}$ disappears modulo $\gamma_{i+2}(G)$.
    And $[x_i, a_j]$ are commutators of length  $i+1$ in the elements of  $M$.
    This proves the Lemma.

    So any element of $\gamma_{i+1}$ can be generated by $i+1$-fold commutators modulo $\gamma_{i+2}$. So $\gamma_i(G)$ is generated by all  left $i$-fold commutators and $\gamma_{i+1}$.
\end{proof}

Let's use now this theorem.


    Let $G$ be finitely generated and nilpotent.
    Consider 
    \[
        G = \gamma_1(G) \ge  \gamma_2(G) \ge  \cdots \gamma_i(G) \ge  \gamma_{i+1}(G) \ge  \gamma_{c+1}(G) = 1
    .\] 
    This implies that $\gamma_i(G) / \gamma_{i+1}(G)$ is abelian, by the construction of the lower central series.
    Now, by the previous lemma, $\gamma_i(G) / \gamma_{i+1}(G)$ is finitely generated.
    If $A$ is a finitely generated abelian group, we can write it as
    \[
    A = C_1 \oplus C_2 \oplus \cdots C_k
    ,\] 
    where $C_i$ is cyclic ($\Z_p$,  $C_{\infty}$).
    Then we can make a sequence
    \[
    A = A_1 \ge  A_2 \ge  A_3 \ge  \cdots A_{k+1} = 1
    ,\] 
    where $ A_1$ is the total group, $ A_2 = C_2 \oplus \cdots \oplus C_k$, $ A_3 = C_3 \oplus \cdots \oplus C_k$, \ldots
    Then $A_i / A_{i+1}$ is cyclic.

    We can apply this to our series.
    Indeed, fix $i$. Then $\gamma_i / \gamma_{i+1}$ is finitely generated and abelian. So we can find a series of subgroups
    \[
        \frac{\gamma_i}{\gamma_{i+1}} =
        \frac{A_1}{\gamma_{i+1}(G)} \ge  
        \frac{A_2}{\gamma_{i+1}(G)} \ge  
        \frac{A_3}{\gamma_{i+1}(G)} \ge   \cdots
        \frac{A_{k+1}}{\gamma_{i+1}(G)}  = 1
    .\] 
    such that $A_j / A_{j+1}$ is cyclic.
    Plugging these groups inside the original lower central series.
    Then we have

    \[
        \cdots \ge  \gamma_{i}(G) = A_1 \ge  A_2 \ge  \cdots \ge  A_{k+1} = \gamma_{i+1} \ge  \cdots
    .\] 
    We have refined our series,
    in such a way that any quotient is cyclic.

    We have found the following:

\begin{prop}[17.2.2, first part]
    If $G$ is a finitely generated nilpotent group, then $G$ has a central series of the form
    \[
        G = G_1 \ge  G_2 \ge  \cdots \ge  G_n  = 1,
    \]
    with $G_{i} / G_{i+1}$ is cyclic.
\end{prop}

\begin{prop}[17.2.2, second part]
    TODO \ldots
\end{prop}

\begin{proof}
    Not examinable.
\end{proof}

Tip: read the rest of the chapter. $17.2.5$ interesting. Understand statement, useful!


\part{Polycyclic groups}
\renewcommand{\thesection}{\Alph{section}}

% \nsection{1}{1}{Max condition and solvable groups}

 \begin{definition}
    A group $G$ has the max condition iff one of the following $3$ equivalent conditions hold:
    \begin{enumerate}[(a)]
        \item Every family of subgroups of $G$ has a maximal member (w.r.t.\ inclusion
        \item Any strictly increasing series of subgroups has to be finite.
        \item Any subgroup of $G$ is finitely generated. In particular, $G$ is finitely generated.
    \end{enumerate}
\end{definition}
\begin{proof}
    $(a) \implies (b)$ trivial.
    $(b) \implies (a)$ is Zorn's Lemma.

    Suppose $(b)$ holds.
    Let  $H \le  G$. Then we have to show that $H$ is finitely generated. Suppose  $H \neq  1$ and pick $h_1 \in H$ and $H_1 = \left<h_1 \right>$.
    Let $h_2 \in  H \setminus H_1$ and consider $ H_2 = \left<h_1, h_2 \right>$.
    Continuing, we have
    \[
    H_1 < H_2 < H_3 < \cdots 
    ,\] 
    which is by $(b)$ finite, so  $H_k = H$, so  $H$ is finitely generated.

    Converse.  $(c) \implies (b)$.
    Consider
    \[
    H_1 < H_2 < H_3 < \cdots
    .\] 
    Let $H = \bigcup H_i \le G$.
    $H$ is finitely generated, so  $H = \left<a_1, \ldots, a_k \right>$
    then $ a_1 \in H_{i_1}$, $ a_2 \in H_{i_2}$, so $a_i \in H_m$ where is $\max \{i_1, \ldots, i_k\} $.
    So $H_m = H$.
\end{proof}

\begin{prop}
    Let $G$ be a group and $N \triangleleft G$, then $G$ has max iff $N$ and $G / N$ have max.
\end{prop}
\begin{proof}
    Assume that $G$ has max.
    If $H \le N$, then $H \le  G$ and so fg.
    If $K \le  G / N$ then $K$ is of the form $L / N \le  G / N$ for some $L \le G$. Then $L$ is fg because $G$ is max, so $L / N$ is fg.

    Suppose $N$ and $G / N$ have max.
    Let  $H \le G$.
    First, $H \cap N$ is fg, because subgroup of $N$. 
    Secondly, $\frac{H N}{N}$ is fg.
    The first gives rise to  generateors $ a_1, \ldots, a_n$ and the second $ b_1N, \cdots b_k N$ with $b_i \in H$.
    Then 
    \[
    H = \{a_1, \ldots, a_n, b_1, \cdots, b_k\} 
    .\] 
\end{proof}

\begin{eg}[Groups having max]
    \begin{enumerate}
        \item Finite groups
        \item Cyclic groups
    \end{enumerate}
    Any group build up from $(1)$ and $(2)$ have max.
    For example,  $ \Z^{k}$ is max.
    Indeed, $ \Z \le \Z^{k}$ has max, $ \Z^{k} / \Z = \Z^{k-1}$ has max by induction \ldots     

    There are groups having max not fitting into this scheme. 
\end{eg}
\begin{eg}[A group that is fg that has not max]
    Consider
    \[
        \bigoplus_{z \in \Z} \Z \ni (\cdots, a_{-1}, a_0, a_1, \cdots)
    ,\] 
    with almost all of them zero.
    Then consider $t \in \Aut(\bigoplus_{\Z} \Z)$ that is just shift to the right.
    Then consider
    \[
        \bigoplus_{\Z} \Z \rtimes \Aut\left(\bigoplus_{z \in \Z} \Z\right) = \Hol\left(\bigoplus_{z \in \Z} \Z\right)
    .\] 
    And consider the subgroup
    \[
    G = \left<\bigoplus_{z \in \Z} \Z, t \right>
    .\] 
    Claim: $G$ is generated by
    \[
        s = (\cdots, 0, 0, 1, 0, 0, \cdots) \text{ and } t
    .\] 
    Indeed, $t s t^{-1} = t(s)$.
    But $\bigoplus \Z \le  G$, but not finitely generated.
\end{eg}

\begin{definition}
    Let $P$ be a property of groups.
    Then a group $G$ is said to by poly-$P$ iff there is a sequence
    \[
    1 = G_0 \triangleleft G_1 \triangleleft  G_2 \triangleleft  \cdots G_n = G
    \] 
    such that $G_{i+1} / G_{i}$ have property $P$.
    (Note: not required that $G_i \triangleleft G$.)
\end{definition}

\begin{definition}
    If $P$ and $Q$ are properties of a group.
    Then $G$ is said to be $P$-by-$Q$ iff
    $G$ has a normal subgroup $N \triangleleft  G$ such that
    $N$ has property  $P$ and  $G / N$ has property  $Q$.
\end{definition}
