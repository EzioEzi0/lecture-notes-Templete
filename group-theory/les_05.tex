\lesson{5}{di 13 okt 2020 11:28}{}

\begin{remark}
    If $\rho$ and $\sigma$ are two equivalent representations. So there is a $G$-map between them. So their matrices are in the right bases the same, so 
    \[
        \chi_{\rho} = \chi_{\sigma}
    .\] 

    In other words:
    \[
        \begin{tikzcd}
            V \arrow[d, "\rho(g)"]\arrow[r, "A"] & W \arrow[d, "\sigma(g)"]\\
            V \arrow[r, "A"] & W
        \end{tikzcd}
    \]
    So then
    \[
        \rho(g) = A^{-1} \sigma(g) A
    ,\] 
    so 
    \[
        \Tr \rho(g) = \Tr \sigma(g)
    .\] 

    The remarkable thing is the reverse is also true!
\end{remark}

First a corollary of the previous result.

\begin{corollary}
    Let $\rho: G \to \GL(V)$ and $\sigma: G \to \GL(W)$ be irreducible representations.
    Then 
    \[
        (\chi_{\rho} | \chi_{\sigma}) = \begin{cases}
            0 & \text{if $\rho$ is not equivalent to $\sigma$}\\
            1 & \text{if $\rho$ is equivalent to $\sigma$.}
        \end{cases}
    \] 
\end{corollary}
\begin{proof}
    \begin{itemize}
        \item Suppose $\rho$ is not equivalent to $\sigma$.
            Then $\Hom_{G}(V, W)$.
            Because $\rho$ and  $\sigma$ are irreducible, Schur's Lemma says that any  $G$-map is either $0$ or an isomorphism. But $\rho$ and $\sigma$ are not equivalent, so there are no isomorphism. So  $ \Hom_{G}(V, W) = 0$.
            This then means that $(\chi_{\rho} | \chi_{\sigma}) = \dim \Hom_{G}(V, W) = 0$.
        \item Suppose that $\rho$ is equivalent to $\sigma$.
            Then their characters are the same, so 
            \begin{align*}
                (\chi_{\rho} | \chi_{\sigma})_G &= (\chi_{\rho} | \chi_{\rho})_G = \dim_G \Hom_{G}(V, V)
            .\end{align*}
            Then we again use Schur's Lemma. Any $G$-map from $V \to  V$ is multiplication by a scalar (our vector spaces are over $\C$).
            So therefore $ \Hom_{G}(V, V) \cong \C ,$ 
            so
            \[
                (\chi_{\rho} | \chi_{\sigma})_G = 1
            .\] 
    \end{itemize}
\end{proof}

\begin{remark}
    Let $\rho_1, \ldots, \rho_r$ be irreducible representations of $G$ such that $\rho_i \not\cong \rho_j$ if  $i \neq j$.
    Then we have that $(\chi_{\rho_i} | \chi_{\rho_j} ) = \delta_{ij}$.
    So these are orthonormal vectors of space $C(G)$. 
    This implies that $r \le  \dim C(G)$. So we cannot find more irreducible representation (up to equivalence) than the dimension of $C(G)$, the vector space of class functions.
    Remember, this dimension was the number of conjugacy classes of $G$.
    So  $r \le  \text{\# cong. classes of $G$}$.
\end{remark}

\begin{definition}[Irreducible character]
    If $\rho$ is an irreducible representation, then $\chi_{\rho}$ is an irreducible character.     
\end{definition}

\begin{theorem}[3.15]
    Let $\chi_{1}, \ldots, \chi_{r}$ be all the different irreducible characters of $G$\footnote{Note that we don't have to specify `up to equivalence', because equivalent representations give rise to the same character}
    By the previous discussion,  $r < \dim C(G)$. But even more:  $\{\chi_{1}, \chi_{2}, \ldots, \chi_{r}\} $ is a basis of $C(G)$. 
    As a result: $r = \dim C(G)$.
\end{theorem}
\begin{proof}
    The only thing we have to show is that these vectors generate the whole space.
    Let $\alpha$ be a class function of $G$.
    suppose that $(\alpha | \chi_{_i}) = 0$ for all $i \in \{1, 2, \ldots, r\} $
    We claim that $\alpha \equiv 0$.
    This will prove that  $\{\chi_{_i}\}^{\perp}  = \O$, proving the theorem.

    (By the way, note that we can at least build all characters (also the ones which are not irreducible), because the character of a reducible representation is the sum of the characters of the irreducible components. But we want to show we can create \emph{any} class function)

    Pick any representation $\rho: G \to \GL(V)$.
    Define
    \begin{align*}
        \rho_\alpha: V &\longrightarrow V \\
        v&\longmapsto  \sum_{g \in G} \underbrace{\alpha(g)}_{\in \C} (\rho(g)(v))
    .\end{align*}
    We claim that this map is a $G$-map.

    \begin{proof}
        Pick any $h \in G$.
        Then
        \begin{align*}
            \rho_\alpha(\rho_h v) &= \sum \alpha(g) \rho(g)(\rho(h)(v))\\
                                  &= \sum \alpha( g ) \rho(h) \rho(h)^{-1} \rho(g) \rho(h) v\\
                                  &= \sum \alpha(g) \rho(h) \rho(h^{-1} gh) v
                                  \shortintertext{Because $\alpha$ is a class function we have}
                                &= \sum \alpha(h^{-1}g h) \rho(h) \rho(h^{-1} gh) v\\
                                &= \rho(h) \sum_{g \in G} \alpha(g) \rho(g) v\\
                                &= \rho(h) \rho_\alpha(v)
        .\end{align*}
    \end{proof}

    Now we proved this for any representation. Now we will pick an irreducible one.
    Let $\rho$ be an irreducible representation. Then  $\rho_\alpha \in  \Hom_{G}(V, W)$. So $\rho_\alpha = \lambda 1_V$ by Schur's lemma.
    This implies that $\Tr \rho_\alpha = \lambda \dim V$.
    We will show that $\rho_\alpha$ is zero. 
    Indeed, the trace is also
    \begin{align*}
        \Tr(\rho_\alpha) &= \sum_{g \in G} \alpha(g) \Tr(\rho(g)) \\ 
                         &= \sum_{g \in G} \alpha(g) \chi_{\rho}(g)\\
                        &= |G| \frac{1}{|G|} \sum_{g \in G} \alpha(g) \chi_{\rho}(g)\\
                        &= |G| \frac{1}{|G|} \sum_{g \in G} \alpha(g) \overline{\overline{\chi_{\rho}(g)}}\\
                        &= |G| \frac{1}{|G|} \sum_{g \in G} \alpha(g) \overline{\chi_{\rho^*(g)}}\\
                        &= |G| ( \alpha | \chi_{\rho^{*}} )_G
                        \shortintertext{Now, Problem I: if $\rho$ is irreducible then $\rho^*$ is irreducible. (But you don't really need it here, because we assumed this was zero for any representation \ldots)}
                        &= |G| 0 = 0
    .\end{align*} 
    This then implies that $0 = \lambda \dim V$, so  $\lambda = 0$, hence $\rho_\alpha$ = 0.

    So what we did was the following: Suppose that $(\alpha | \chi_{i}) = 0$ for all $i$ and all  $\alpha$.
    Then assuming that  $\rho$ was irreducible, then we showed that $\rho_\alpha$ is the zero map.

    Now assume that $\rho = \rho_1 \oplus \rho_2 \oplus \cdots \oplus \rho_k$ with $\rho_i$ irreducible. So  $V = V_1 \oplus V_2 \oplus \cdots \oplus V_k$.
    Then it's easy to see that
    \[
        \rho_\alpha = (\rho_1)_\alpha \oplus (\rho_2)_\alpha \oplus \cdots \oplus (\rho_k)_\alpha
    .\] 
    Because $(\rho_i)_\alpha = 0$, so also $\rho_\alpha = 0$.
    So this is also true for reducible representations.

    \hr
    Let us now consider the regular representation
    \[
        \rho_\text{reg}  \to  \GL(\C[G]) : g \mapsto  (a h \mapsto a (gh))
    .\] 

    Net let $v = 1 e \in \C[G]$.
    Then
    \begin{align*}
        0 = (\rho_\text{reg})_\alpha (e) &= \sum_{g \in G} \alpha(g) \rho_\text{reg}(g) e\\
                                         &= \sum_{g \in G} \alpha(g) g
    .\end{align*}
    This implies that for all $g \in G$ that $\alpha(g) = 0$, so  $\alpha \equiv 0$.

    This proves the claim. 
    Why?
    Consider any $\alpha \in C(G)$.
    Consider
    \[
        \alpha' = \alpha - \sum_{i=1}^{r}(\alpha | \chi_{i}) \chi_{i}
    .\] 
    This is a class function, so $\alpha' \in C(G)$.
    Then 
    \begin{align*}
        (\alpha' | \chi_{k})
        &= (\alpha | \chi_{k}) - \sum (\alpha | \chi_{i}) (\chi_{i} | \chi_{k})\\
        &= (\alpha | \chi_{k}) - \sum (\alpha | \chi_{i}) \delta_{ij}\\
        &= (\alpha | \chi_{k}) - (\alpha| \chi_{k})\\
        & = 0
    .\end{align*} 
    Conclusion: $\alpha' = 0$ so  $\alpha = \sum_{i=1}^{r} (\alpha | \chi_{i}) \chi_{i}$
\end{proof}

So we proved that for each conjugacy class, there is a irreducible representation. But we still don't know how we can find it.


\begin{corollary}[Morse or less Theorem 3.16]
    Let $V$ be a $G$-module. Suppose that $V = V_1 \oplus V_2 \oplus \cdots \oplus V_m$ where each $V_i$ is a irreducible submodule.
    Let $\chi_{1}, \chi_{2}, \ldots \chi_{r}$ be all the different irreducible characters with corresponding representations $\rho_i$.
    Let $\rho$ be the representation corresponding to the  $G$-module $V$.
    Then $(\chi_{\rho} | \chi_k)$ is equal to the number of modules $V_i$ which are  $G$-equivalent to $\rho_k$.
\end{corollary}

So this kinda proves that the decomposition of $V$ is unique upto equivalence of submodules. Indeed $(\chi_{\rho} | \chi_{k})$ does not depend on the decomposition and gives you the number of modules in the decomposition that are equivalent to $\rho_k$.

\begin{proof}
    Let $\sigma_i: G \to \GL(V_i)$ denote the representation corresponding to $V_i$.
    Then  $\rho = \sigma_1 \oplus \cdots \oplus \sigma_n$.
    Then $\chi_{r} = \chi_{\sigma_1} + \cdots + \chi_{\sigma_n}$.
    So $(\chi_{\rho} | \chi_{k}) = \sum (\chi_{\sigma_i}  | \chi_{k})$.
    Note that $\sigma_i$ and  $\rho_k$ are irreducible so their inner product is $0$ or $1$ iff they are equivalent or not. So
     \[
         (\chi_{\rho} + \chi_{_k}) = \underbrace{1 + \cdots + 1}_{\ell  \text{ times }} = \ell
    ,\] 
    where $\ell$ is exactly what we were looking for.
\end{proof}

Let's discuss the latest topic of representation theory: character tables.


\nsection{3}{4}{Character tables}

For a finite group $G$, its character table if of the form


\begin{tabular}{ccccc}
         & $c_1$ & $ c_2$ & $\cdots$ & $c_r$ \\ \hline
    $\chi_1$ &  & & & \\
    $\chi_2$ &  & & & \\
    $\vdots$ &  & & $\chi_{i} (c_j)$ & \\
    $\chi_r$ &  & & & \\
\end{tabular}

\begin{itemize}
    \item Here, we have $c_i \in  C_i$ where $C_i$ a conjugacy class. So for each column we have a representative of each conjugacy class.
Note: by convention: $C_1 = \{e\}$, so $ c_1 = e$.

\item The $\chi_i$ are the different irreducible characters.
Convention: $\chi_1$ corresponds to the trivial  $1$-dimensional representation. Therefore $\chi_1(g) = 1$ for all $g$.

\item Note that $\chi_{i}(e) = \text{dimension of the $i$-representation}$. So the first column contains the dimension of the representation.
    Convention: increasing dimension.

\end{itemize}

\begin{eg}
    Let $G = S_3 = D_3 = \{1, a, a^2, b, ab, a^2b\} $. with $a^3 = 1$, $b^2 = 1$ and $ba = a^2 b$.

    What are conjugacy classes?
    \begin{align*}
        c_1 &= \{1\} \\
        c_2 &= \{a, a^2\} \\
        c_3 &= \{b, a^2b, ab\} 
    .\end{align*}

    So our table looks as follows:

    \begin{center}
    \begin{tabular}{cccc}
        & $1$ &  $a$ &  $b$ \\ \hline
        $\chi_{1}$ & 1& 1 & 1\\
        $\chi_{2}$ & & \\
        $\chi_{3}$ & & \\
    \end{tabular}
    \end{center}

    In the exercises, we saw that if $N \triangleleft G$, then $\rho: G / N \to  \GL(V)$ is a representation as well.

    Here, $\{1, a, a^2\} \triangleleft D_3$ and we have that $ \frac{D_3}{\{1, a, a^2\} } = \Z_2 = \Z / 2\Z$.
    Now, there's an easy representation of $\Z_2$:
    \begin{align*}
        \rho: \Z_2 &\longrightarrow \GL(C)  \\
        t &\longmapsto -1\\
        1 &\longmapsto 1
    .\end{align*}
    So now, we have for $D_3$: something that maps $a$ to $1$ and  $b$ to  $-1$.

    So we have in our table:

\begin{center}
    \begin{tabular}{cccc}
        & $1$ &  $a$ &  $b$ \\ \hline
        $\chi_{1}$ & 1& 1 & 1\\
        $\chi_{2}$ &$1$ & $1$& $-1$\\
        $\chi_{3}$ & & \\
    \end{tabular}
    \end{center}

    For the last one, remember that $D_3$ was isometries from triangle in the plane mapping to itself.
    So we have a representation in the plane:
    \[
    a = \begin{pmatrix}
        \cos 2 \pi / 3 & \sin 2 \pi /3 \\
        -\sin 2 \pi /3  & \cos 2 \pi / 3 \\
    \end{pmatrix}
    \qquad b = \begin{pmatrix}
        1 & 0 \\
        0 & -1
    \end{pmatrix}
    .\] 
    So $\Tr a = -1$ and  $\Tr b  = 0$. We claim that this is an irreducible representation. So the character table becomes

    \begin{center}
    \begin{tabular}{cccc}
        & $1$ &  $a$ &  $b$ \\ \hline
        $\chi_{1}$ & 1& 1 & 1\\
        $\chi_{2}$ &$1$ & $1$& $-1$\\
        $\chi_{3}$ & 2& $-1$& 0\\
    \end{tabular}
    \end{center}

\end{eg}


\begin{prop}
    Let $\rho: G \to \GL(V)$ be a representation with character $\chi_\rho$.
    Then  $\rho$ is irreducible iff $(\chi_{\rho} | \chi_{\rho})_G = 1$
\end{prop}
\begin{proof}
    Let $\chi_1 \chi_{2}, \ldots, \chi_{r}$ be all the different irreducible characters.
    Then
    \[
    \chi_{\rho} = \sum n_i \chi_{i}
    .\] 
    Then the inner product is
    \[
        (\chi_{\rho} | \chi_{\rho}) = \sum \sum n_i n_j (\chi_{i}  \mid  \chi_{j})_G = n_1^2 + n_2^2 + \cdots + n_r^2
    .\] 
    This is a sum of natural numbers. (Q: why not complex?)
    This is one if exactly one of the $n_i = 1$ and the others are  $0$.
\end{proof}

\begin{eg}[Previous example, continuoud]
    \[
        (\chi_{3} | \chi_{3}) = \frac{1}{6} (
        \chi_{3}^2(e) + 
        \chi_{3}^2(a) + 
        \chi_{3}^2(a^2) + 
        \chi_{3}^2(b) + 
        \chi_{3}^2(ab) + 
        \chi_{3}^2(a^2b)
        ) = 1
    .\] 
    We should also check that the inner product with the other ones is $0$.
\end{eg}
