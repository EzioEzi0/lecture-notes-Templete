\lesson{14}{di 10 nov 2020 13:09}{}
\part{Homological Algebra and (co)homology}

\nchapter{1}{Basic homological algebra}
\nsection{1}{1}{Projective, Injective and flat modules}

$R$ is a ring (non-commutative!) with $1$.

\begin{definition}
    A sequence of $R$-modules over $R$ 
    \[
    \cdots \to  M_{n+1} \xrightarrow{f_{n+1}}  M_n  \xrightarrow{f_n}   M_{n-1} \to \cdots
    \] 
    is exact at place $n$ if Kernel is Image.
\end{definition}

\begin{eg}
    $0 \to  A \xrightarrow{f}  B $ is exact iff $f$ is injective.
\end{eg}
\begin{eg}
    $A \xrightarrow{p} B \to  0$ is exact iff $p$ is surjective.
\end{eg}
\begin{eg}
    $0 \to  A \xrightarrow{f}  B \to  0$ is exact iff $f$ is an isomorphism.
\end{eg}
\begin{definition}
   A short exact sequence is an exact sequence of the form
   \[
   0 \to  A \xrightarrow{i} B \xrightarrow{p} C \to  0
   .\] 
   $p$ induces an isomorphism $\frac{B}{i(A)} \cong C$.
\end{definition}

\begin{lemma}[1.3]
    For a short exact sequence 
    \[
    0 \to  A \xrightarrow{\alpha}   B  \xrightarrow{\beta}   C \to  0
    ,\] 
    TFAE
    \begin{enumerate}[(1)]
        \item  There exists $\phi: B \to  A$ such that $1_A = \phi  \circ  \alpha$
        \item There exists $\theta: C \to  B$ such that $1_C = \beta  \circ  \theta$ 
        \item  There exists $\psi: B \to  A \oplus C$ that makes the following diagraom commutative:
            \[
                \begin{tikzcd}
                    0 \arrow[r, ""] & A \arrow[r, "\alpha"] \arrow[d, "Id"]& B \arrow[r, "\beta"] \arrow[d, "\psi"] &C \arrow[r, ""] \arrow[d, "Id"]&  0\\
                    0 \arrow[r, ""]& A \arrow[r, ""] & A \oplus C \arrow[r, ""] & C \arrow[r, " "]& 0
                \end{tikzcd}
            \]
            and this $\psi$ will be an isomorphism.
    \end{enumerate}
\end{lemma}
This is something specific for $R$-modules! Does not work for general groups. Uses the fact that underlying group is abelian!

If one (or all) of this conditions is satisfies, then we say that the sequence splits.


\paragraph{Functors}
Let $X$ be a left $R$-module.
Then $\Hom_R(X, -)$ is a covariant functor from the category of left $R$-modules to the category of abelian groups.
If $A$ is another left $R$-module, then
\[
    \Hom_{R}(X, A) = \{f : X \to  A \}
.\] 
is an abelian group (we can add two morphisms). 
Suppose we have $A \xrightarrow{f} B$.
Then $ \Hom_{R}(X, A) \xrightarrow{f_*} \Hom_{R}(X, B)$.
Where $f_*(\alpha) = f  \circ  \alpha$.

Covariant: because if $f : A \to  B$ then derived map also from Hom $A$ to Hom $B$.


Let $Y$ be a left $R$-module.
Suppose $A \xrightarrow{f} B$ then 
\[
\Hom_{R}(B, Y) \xrightarrow{f^{*}}  \Hom_{R}(A, Y)
.\] 
where $f^{*}(\alpha) = \alpha  \circ  f$.

You can show this is a functor:  $\operatorname{Id}$ maps to identity and $(f  \circ g)_* = f_*  \circ  g_*$.


\begin{theorem}
    $ \Hom_{R}(X, -)$ is left exact.
    If  the following is exact,
    \[
    0 \to  A \xrightarrow{f} B \xrightarrow{g}  C
    ,\] 
    then
    \[
    0 \to  \Hom_{R}(X, A) \xrightarrow{f_*} \Hom_{R}(X, B) \xrightarrow{g_*}   \Hom_{R}(X, C)
    \] 
    is also exact.
\end{theorem}
\begin{theorem}
    $ \Hom_{R}(-, Y)$ is left exact.
    If 
    \[
    A \to B \to  C \to  0
    \] 
    is exact, then
    \[
    0 \to  \Hom_{R}(C, Y) \to  \Hom_{R}(B, Y) \to  \Hom_{R}(A, Y)
    \] 
    is also exact.
\end{theorem}

\paragraph{Tensor product}

Let $A$ be a right $R$-module and $B$ be a left $R$-module.
Then $A \otimes_R B$ is defined.
It is the abelian group generated by $a \otimes b$ for all $a \in A, b \in B$
and with relations 
\begin{align*}
    (a + a') \otimes b &= a \otimes b + a' \otimes b\\
    a \otimes (b + b') &= a \otimes b + a \otimes b'\\
    a r\otimes b &= a \otimes rb
.\end{align*} 
Note that in the last relation, we multiply $a$ on the right and $b$ on the left.

Suppose we  would do this for $A$ left module $R$.
So then $ ra \otimes b = a \otimes rb .$ 
But then we would have a problem:

\[
    (r s a) \otimes b = a \otimes (rs)b
.\] 
but
\[
    (rsa) \otimes b = s a \otimes rb = a \otimes (sr) b
.\] 
But ring is not necessary commutative, so this is a problem.

\paragraph{Tensor functors!}

Let $A$ be a right $R$-module, then $A \otimes_R -$ will be a covariant functor from the category of left $R$-modules to the category of abelian groups.

Suppose $B \xrightarrow{f}  C$.
Then this induces a map 
\[
    A \otimes_R B \to  A \otimes_R C: a \otimes b \mapsto  a \otimes f(b)
.\] 

We can do the same thing with a left module:
Suppose $B$ is a left module.
Then $- \otimes_R B$ will be a covariant functor from the category of right $R$-modules to the category of abelian groups.

\begin{prop}
    Fact: both $A \otimes_R -$ and $- \otimes_R B$ are right-exact.
    \begin{eg}
        $C \to  D \to  E \to  0$ is a s.e.s of left $R$-modules, then
        \[
        A \otimes C \to  A\otimes D \to  A\otimes E \to  0
        \] 
        is also exact.
    \end{eg}
\end{prop}

\begin{definition}
    A functor $T$ is exact iff for any s.e.s.
    \[
        0 \to  A \to  B \to  C \to  0
    ,\]  we have that the sequence
    \[
        0 \to  T(A) \to  T(B) \to  T(C) \to  0
    \] 
    is also exact.
\end{definition}

\begin{definition}
    A left module $P$ is projective iff for any surjective $R$-map $p: A \to  A''$ and $R$-map $h: P \to  A''$ there is an $R$ map $g: P \to  A$ making the diagram commute
    \[
        \begin{tikzcd}
            & P \arrow[d, "h"] \arrow[dl, dashed, swap,"g"]\\
            A \arrow[r, "p"] & A'' \arrow[r, ""] & 0
        \end{tikzcd}
    \]
\end{definition}

\begin{prop}
    The following are equivalent
    \begin{enumerate}[(1)]
        \item  $P$ is a projective module
        \item Every s.e.s $0 \to  A \to  B \to  P \to  0$ splits.
        \item  There exists $Q$ an  $R$-module such that $Q \oplus P$ is  a free $R$-module.
        \item $\Hom_R(P, -)$ is an exact functor.
    \end{enumerate}
\end{prop}
\begin{proof}
    Hints on how to prove this.
    \begin{itemize}
        \item [$1 \implies 2$]. Consider $h:P \xrightarrow{\operatorname{Id}}   P$ and then $g$ is a section of  $p$. 
        \item [$2 \implies 3$] Note that any $R$-module is the quotient of a free module $F$.
            $P \cong F / Q$.
            Then we have s.e.s.
             \[
            0 \to  Q \to  F \to  P \to  0
            .\] 
            This splits because of $2$, so  $F \cong Q \oplus P$
    \end{itemize}
\end{proof}

\begin{remark}
    A free module is a projective module.
\end{remark}

\begin{definition}
    A left module $I$ is injective if

    \[
        \begin{tikzcd}
            & I\\
            0 \arrow[r, ""] & A\arrow[u, "f"] \arrow[r, " "] & B \arrow[ul, dashed, "g"]
        \end{tikzcd}
    \]
\end{definition}

\begin{prop}
    \begin{enumerate}[(1)]
        \item $I$ is an injective module
        \item Every s.e.s. $0 \to  I \to  B \to  C \to  0$ splits
        \item The functor $ \Hom_{R}(-, I)$ is an exact functor.
    \end{enumerate}
\end{prop}

Note: no mention of being part of a free module.


\begin{definition}
    A right module $A$ (left  $R$-module  B)
    is flat iff $ A \otimes_R - $ (resp $- \otimes_R B$) is an exact functor.
\end{definition}
\begin{theorem}
    Free modules are also flat.
\end{theorem}
