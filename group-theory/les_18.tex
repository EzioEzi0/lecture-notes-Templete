\lesson{18}{di 24 nov 2020 13:26}{}
\url{https://www.youtube.com/watch?v=1TOWZbJjw0U}

\section{Tor functors}

Let $G$ be a left $R$-module.
Then $T = - \otimes_R G$ is an additive covariant functor from right $R$-modules to abelian groups.
Then
\[
    \Tor_n^{R}(A, G) = (L_n T)(A)
.\] 

Let $0 \to  A' \to  A \to  A'' \to  0$ an exact sequence.
Then this induces a long exact sequence of $\Tor$:
\[
    \Tor_n^{R}(A, G) \to  \Tor_n^{R}(A'', G) \to  \Tor_{n+1}^{R}(A', G) \to  \cdots
.\] 
So because covariant, arrows in same direction.

\begin{remark}
    We can also start with the functor $A \otimes_R -$, which is covariant.
    Then we also get derived functors $\operatorname{tor}_n^{R} = H_n(A \otimes_R Q_*)$ where $Q_*$ is the projective resolution of $G$
\end{remark}
\begin{theorem}[Big theorem!]
    For any modules $A$ and $G$, 
    \[
        \Tor_n^{R}(A, G) \cong \operatorname{tor}_n^{R}(A, G)
    .\] 
\end{theorem}
\begin{proof}
    Not easy!
\end{proof}

So we can choose what module we use the projective resolution of.

\begin{prop}[2.16]
    $\Tor_0^{R}(A, B) = A \otimes_R B$.
\end{prop}
\begin{proof}
    Because tensor is left exact functor.
\end{proof}

\begin{theorem}[2.18]
    If $P$ is a projective right $R$-module.
    If $Q$ is a projective left  $R$-module.
    $A$ is a right $R$-module.
    $B$ is a left $R$-module.
    Then for any $n\ge 1$,
    \[
        \Tor_n^{R}(P, B) = 0 \qquad \text{and} \qquad
        \Tor_n^{R}(A, Q) = 0
    .\] 
\end{theorem}
\begin{remark}
    The other way around does not hold. (For Ext it does!)
\end{remark}

Why name? 

Consider abelian groups, so $\Z$-modules.

\begin{eg}
    Let $A$ be any abelian group. Then
    \[\Z_m \otimes A = \Tor_0^{\Z} (\Z_m, A) = \frac{A}{mA}\]

    And
    \[
        \Tor_{1}^{\Z}(\Z_m, A) = \{a \in A  \mid  m a = 0\},
    \] 
    which is the set of $m$-torsion elements of $A$.
    \[
        \Tor_n^{\Z}(\Z_m, A) = 0 \qquad n > 1
    .\] 
\end{eg}
\begin{proof}
    Projective resolution of $\Z_m$:
    \[
        0 \to  0\to  \Z \xrightarrow{\times m}   \Z \to  \Z_m \to  0
    .\] 
    So deleteing an tensoring,
    \[
        0 \to  \Z\otimes A \xrightarrow{(\times m) \otimes 1_A}   \Z \otimes A \to  0
    .\] 
    Now, $\Z \otimes A\cong A$.

    \[
        a \mapsto 1 \otimes a \mapsto  m \otimes a = m a \otimes 1 \mapsto ma
    ,\] 
    so the map $A \to  A$ is also $\times  m$
    Then zeroth homology is kernel which is everything divided out by kernel, so indeed $\frac{A}{mA}$.
    The first tor, elements that get mapped to zero under $\times  m$, so indeed $\{ a \in A  \mid  m a = 0\} $
\end{proof}

\begin{remark}
    $\Tor_1(\frac{\Q}{\Z}, A)$ is the torsion subgroup of $A$.
    Not so easy, but not so difficult.
     $\Q / \Z$ contains all groups $\Z_n$ for all $n$.
\end{remark}

\nchapter{3}{Cohomology of groups}
\nsection{3}{1}{$G$-modules}
\begin{definition}
Let $G$ be any group.
By a $G$-module, we mean an abelian group $A$ together with an action
 \[
     G  \times  A \to  A (g, a) \mapsto  ga
,\] 
such that $g(a+b) = ga + gb$,  $g_1(g_2(a)) = (g_1g_2)(a)$ and $1a = a$.
\end{definition}

Equivalently, this action is given by a homomorphism $\rho: G \to  \Aut A$.\footnote{In representation theory, we did $\GL_n(A)$}

But we want modules over a ring.
So we consider the groupring $\Z G = \Z[G]$.
Elements are finite sums $\sum_{g \in G} n_g g$

There is a one to one correspondence between $G$-modules  and $ \Z G$-modules.


The following is a ring morphism:
\begin{definition}[Augmentation map]
    \[
    \epsilon: \Z G \to  \Z: \sum_{g \in G} n_g g \mapsto  \sum_{g \in G} n_g
    \]
\end{definition}
\begin{definition}[Augmentation ideal]
    $I = IG = \mathcal G =: \Ker \epsilon$
\end{definition}
This is a subgroup. How does it look?
As an abelian group, $ \Z G$ is just the free abelian group on $G$.

\begin{prop}[3.3]
    $IG$ is the free abelian group generated by  $\{g - 1  \mid  g \in G \setminus \{1\} \} $
\end{prop}
\begin{proof}
    Let $x = \sum n_g g \in IG$.
    So $\sum n_g = 0$.
    So 
    \[
        x = \sum n_g g - \sum n_g 1 = \sum n_g (g-1) = \sum_{g \in G \setminus \{ 1\} } n_g (g-1)
    .\] 
    $IG$ is generated by the set of all $g-1$'s.

    Assume  $\sum_{g \in G\setminus 0} n_g(g-1) = 0$, 

    \begin{align*}
        \sum_{g \in G \setminus 1} n_g g + (- \sum n_g) 1 = 0
    \end{align*}
    but group is free so all $n_g = 0$.
\end{proof}

Every Ideal $I \subset R$ is also an $R$-module.
Also $IG$ is a  $\Z G$-module.

\begin{prop}
    Let $G$ be a group generated by $\{g_1, \ldots, g_n\} $ (actually this alsow orks for infinitely generated groups.)
    Then $IG$ as a  $\Z G$ module is generated by all elements $\{g_1-1, g_2-1, \ldots, g_n - 1 \}$.
\end{prop}

\begin{proof}
It is enough to show that for any $g \in G \setminus \{ 1\} $    , the element $g - 1$ belongs to the $ \Z G$ module generated by 
\[
\{g_1 -1, g_2-1, \cdots, g_n-1\} 
.\] 
Let $g \in  G$, then $g = u_1 u_2 \cdots u_t$ where each $u_j$ is of the form  $g_i^{\pm 1}$.
\begin{align*}
    g - 1 &= u_1 \cdots u_t - 1\\
          &= u_1 u_2 \cdots u_{t-1}(u_t - 1) + u_1 u_2 \cdots u_{t-1} - 1\\
          &= \cdots\\
          &= u_1 u_2 \cdots u_{t-1}(u_{t}-1) + \cdots + u_1 (u_2-1) + 1 (u_1 -1)
.\end{align*}
These are almost of the form we want.
Now, $u_j - 1 = g_i -1 $ or maybe  $g^{-1}_i - 1$.
In the first case, okay.
Second cast $g_i ^{-1}(1-g_i) = (- g_i^{-1})(g_i -1)$.
\end{proof}


\begin{definition}
    Let $A$ be a $G$-module.
    Then 
    \[
    A^{G} = \{a  \in  A  \mid  g a = a \quad \forall g \in g\} 
    .\] 
    This is the largest submodule on $A$ on which $G$ is acting trivially.

    \end{definition}
    \begin{definition}
    \[
        A_G = \frac{A}{(IG) A}
    ,\] 
    which is $A$ divided out by submodule of  $A$ generated by all elements of the form  $ga - a$.
    We call this the coinvariants of $G$.
\end{definition}

\begin{remark}
    $\frac{A}{(IG)A}$ is a trivial $G$-module.
    The induced action will be trivial.
    This is the largest quotient of $A$ which is trivial  $G$-module.
\end{remark}

\nsection{3}{2}{Homology and cohomology of groups}

\begin{definition}
    Let $A$ be a $G$-module and consider $\Z$ as a trivial $G$-module  ($g\cdot z = z$ for all $z$ and $g$).
    The $n$-th cohomology of $G$ with cofficients in  $A$ 
     \[
         H^{n}(G, A) = \Ext_{\Z G}^{n} (\Z, A)
    .\] 
    The $n$-th homology is
    \[
        H_n(G, A) = \Tor^{\Z G}_n(\Z, A)
    .\] 
\end{definition}

\begin{remark}
If $G$ is not abelian then $\Z G$ is not commutative.
\end{remark}
