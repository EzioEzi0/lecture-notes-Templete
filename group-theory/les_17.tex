\lesson{17}{di 24 nov 2020 10:38}{}
\url{https://www.youtube.com/watch?v=2r5-l0cuEwA}

Starts with recap about derived functors.

\begin{definition}
$T$ is a covariant functor from $R$-modules to $S$-modules.
For a given $R$-module $A$, consider the projective resolution
\[
\to  P_3 \to  P_2 \to  P_1 \to  P_0 \to A  \to  0
.\] 
Discard $A$ and apply $T$ :
\[
    \to  TP_3 \to  TP_2 \to  TP_1 \to  TP_0 \to  0
.\] 
Then the left derived functor is
\[
    (L_n T)(A) = H_n(TP_A)
.\] 
\end{definition}
The left derived functor applied to a map $A \to  B$ is defined as follows. Write down projective resolution of $A$ and $B$. We can find maps $f_i: P_i \to  P_i'$. Then $(L_n T)(f) = T(f_n)_*$. 
These maps $f_i$ are not unique but unique upto homotopy, so we have same induced maps.

\begin{lemma}[2.5, Horsehoe lemma]
    Suppose we have a projective resolution of $A$ and $A''$
    \[
        \begin{tikzcd}
            & \vdots \arrow[d, ""] & & \vdots \arrow[d, ""] &\\
            & P_0 \arrow[d, ""] & & P_0'' \arrow[d, ""]&\\
            0 \arrow[r, ""] & A'\arrow[d, ""] \arrow[r, " "] & A \arrow[r, " "] &A''\arrow[d, ""]   \arrow[r, ""] &0 \\
                            & 0 & & 0 &
        \end{tikzcd}
    \]
    Then we can find a projective solution of $A$ such that the diagram commutes and every row is an exact sequence.
\end{lemma}
\begin{proof}
    We will by induction construct layer by layer.
    Let's start with
    \[
        \begin{tikzcd}
            & P_0' \arrow[d, "\epsilon'"] & & P_0'' \arrow[d, "\epsilon''"]\\
            0 \arrow[r, ""] & A'\arrow[d, ""] \arrow[r, ""] & A \arrow[r, ""] & A'' \arrow[d, ""]\arrow[r, ""] & 0\\
                            & 0 & & 0 &
        \end{tikzcd}
    \]
    How to construct $P_0$ and $\epsilon$?
    Take  $ P_0 = P_0' \oplus P_0''$. Then clearly short exact:

    \[
        \begin{tikzcd}
            0 \arrow[r, ""]& P_0' \arrow[r, ""]\arrow[d, "\epsilon'"] & P_0' \oplus P_0'' \arrow[r, " "]& P_0'' \arrow[d, "\epsilon''"] \arrow[r, ""] & 0\\
            0 \arrow[r, ""] & A'\arrow[d, ""] \arrow[r, "i"] & A \arrow[r, "p"] & A'' \arrow[d, ""]\arrow[r, ""] & 0\\
                            & 0 & & 0 &
        \end{tikzcd}
    \]
    (Note that $ P_0$ is a projective module, because it is a direct sum of projective modules. Reasoning: projective if it is a direct factor of a free module. Direct of free modules is free. So direct sum of projective still direct factor of free. Done)
    Now we use that $ P_0''$ is projective. One of the definitions gives then $\sigma$:
    \[
        \begin{tikzcd}
            0 \arrow[r, ""]& P_0' \arrow[r, ""]\arrow[d, "\epsilon'"] & P_0' \oplus P_0'' \arrow[r, " "]& P_0'' \arrow[d, "\epsilon''"] \arrow[r, ""]\arrow[dl, "\sigma"] & 0\\
            0 \arrow[r, ""] & A'\arrow[d, ""] \arrow[r, "i"] & A \arrow[r, "p"] & A'' \arrow[d, ""]\arrow[r, ""] & 0\\
                            & 0 & & 0 &
        \end{tikzcd}
    \]
    Then we can define
    \[
        \epsilon(x', x'') = i(\epsilon'(x')) + \sigma(x'')
    .\] 
    Show that $\epsilon$ is onto, which follows from $\epsilon'$ and $\epsilon''$ being onto. (Exercise!!)
    By construction, the diagram is commutative.

    Now, we can apply the snake lemma.
    Write  $K_0'$,  $K_0$, $ K_0''$ for the kernels, and the cokernels are zero.
    So snake gives us then that $K_0'' \to  \text{cokernel} = 0$, so we have
    \[
        \begin{tikzcd}
            0 \arrow[r, ""] & K_0' \arrow[d, ""]\arrow[r, ""] & K_0 \arrow[d, ""]\arrow[r, ""] & K_0'' \arrow[d, ""]\arrow[r, ""] & 0\\
            0 \arrow[r, ""]& P_0' \arrow[r, ""]\arrow[d, "\epsilon'"] & P_0' \oplus P_0'' \arrow[d, "\epsilon"]\arrow[r, " "]& P_0'' \arrow[d, "\epsilon''"] \arrow[r, ""]\arrow[dl, "\sigma"] & 0\\
            0 \arrow[r, ""] & A'\arrow[d, ""] \arrow[r, "i"] & A \arrow[d, ""] \arrow[r, "p"] & A'' \arrow[d, ""]\arrow[r, ""] & 0\\
                            & 0 & 0& 0 &
        \end{tikzcd}
    \]
    so we know something about the kernels, which is useful for induction.
    Induction: video.

\begin{figure}[H]
    \centering
    \incfig{diagram}
    \caption{TODO diagram}
    \label{fig:diagram}
\end{figure}
\end{proof}

\begin{theorem}[2.6]
    Let $0 \to  A' \to  A \to A'' \to  0$ be a short exact sequence of $R$-modules.
    Then we obtain a long exact sequence of derived functors
    \[
        \to  (L_n T)(A) \to  (L_nT)(A'') \xrightarrow{\delta}   (L_{n-1}T)(A') \to  \cdots \to  (L_0T)A \to  (L_0 T)(A'') \to  0
    .\] 
\end{theorem}
\begin{proof}
    By the previous lemma, we have a commutative diagram:
    \[
        \begin{tikzcd}
            0\arrow[r, ""]& P_2' \arrow[r, ""] \arrow[d, ""]& P_2 \arrow[r, ""] \arrow[d, ""]& P_2'' \arrow[d, ""] \arrow[r, ""]& 0\\
            0\arrow[r, ""]& P_1' \arrow[r, ""] \arrow[d, ""]& P_1 \arrow[r, ""] \arrow[d, ""]& P_1'' \arrow[d, ""] \arrow[r, ""]& 0\\
0\arrow[r, ""]            & P_0' \arrow[r, ""] \arrow[d, ""]& P_0 \arrow[r, ""] \arrow[d, ""]& P_0'' \arrow[d, ""] \arrow[r, ""]& 0\\
0\arrow[r, ""]            & A' \arrow[r, ""] \arrow[d, ""]& A \arrow[r, ""] \arrow[d, ""]& A'' \arrow[d, ""] \arrow[r, ""]& 0\\
                          & 0 & 0 & 0 &
        \end{tikzcd}
    \]
    Now, add $T$s everywhere and ditch row with $A$.
    Original short exact sequences are still exact after applying $T$.
    $0 \to  A \to  B \to  P \to  0$ always splits.
    \begin{uovt}
    Exercise: If $T$ is an additive functor then it preserves split exact sequences.
    \end{uovt}
    This induces a long exact sequence on homology.
    \[
        \cdots \to  H_n(TP') \to  H_n(TP) \to  H_n(TP'') \to  H_{n-1}(TP') \to  \cdots
    ,\] 
    so by definition of $L_nT$ we have de desired result.

    What are these maps? They are defined as before. In notes: spend too much time on this. (Orally video.)
\end{proof}

\begin{remark}
    If $T$ is a contravariant additive functor, we can do the same thing.
    Let $A$ be an $R$-module. Consider deleted projective resolution of $A$
    \[
    P_3 \to  P_2 \to  P_1 \to  P_0 \to  0
    .\] 
    Applying $T$, we have
    \[
        0 \to  T(P_0) \to  T(P_1) \to  \cdots
    .\] 
    This is a cochain complex.
    Then we have the right derived functor:
    \[
        (R^{n} T)(A) = H^{n}((TP_A)^{n})
    .\] 

If $0 \to  A' \to  A \to  A'' \to  0$ is a short exact sequence of $R$-modules and $T$ is a contravariant functor from $R$-modules to $S$-modules, then there is a long exact sequence going the other way around.
\[
    (R^{n} T)(A'') \to (R^{n}T)(A) \to  (R^{n}T)(A') \to (R^{n+1}T)(A'') \to  \cdots
.\] 
\end{remark}

\section{Ext}
Let $G$ be an $R$-module.\footnote{Annoying notation}
Let $T = \Hom_R(-, G)$ the functor we are fixing.
This is a contravariant functor from $R$-modules to abelian groups.

\begin{definition}
    \[
        \Ext^{n}_R(A, G) = (R^{n}(\Hom_{R}(-, G)))(A)
    .\] 
\end{definition}
How do we ``compute'' $\Ext ^{n}_R (A, G)$?
Pick projective resolution.
\[
P_3 \to  P_2 \to  P_1 \to  P_0 \to  A \to  0
.\] 
Then we have
\[
0 \to  \Hom_{R}(P_0, G) \xrightarrow{d_1^{*}}  \Hom_{R}(P1,G) \xrightarrow{d_2^{*}}  \to  \Hom_{R}(P_2, G) \xrightarrow{d_3^{*}}    \cdots
.\] 
This is a cochain complex.
Then 
\[
    H^{n}(-) = \Ext_R^{n}(A, G) = \frac{\Ker d_{n+1}^{*}}{\im d_n^{*}}
.\] 

\begin{prop}[2.9]
    $\Ext_R^{0}(A, G) \cong \Hom_{R}(A, G)$
\end{prop}
\begin{proof}
    \[
    P_2  \xrightarrow{d_1}  P_1 \xrightarrow{\epsilon} \to  A \to  0
    .\] 
    If we apply $\Hom_{R}(-, G)$ to this, and as we know this functor is left exact, we have
    \[
    0 \to \Hom_{R}(A, G) \xrightarrow{\epsilon^{*}} \Hom_{R}(P_0, G) \xrightarrow{d_1^{*}}  \Hom_{R}(P_1, G)
    .\] 
    So,
    \[
        \Ext_R^{0}(A, G = \Ker d_1^{*} = \im \epsilon^{*} \cong \Hom_{R}(A, G)
    ,\] 
    because injectivity $0 \to $.
\end{proof}

\begin{theorem}[2.10]
    If $0 \to  A' \to  A \to  A'' \to  0$ is a short exact sequence of $R$-modules, we get a long exact sequence
    \[
        \cdots \to \Ext_R^{n}(A, G) \to  \Ext_R^{n}(A', G) \to  \Ext_R^{n+1}(A'', G) \to  \cdots
    .\] 
\end{theorem}
So arrows in other direction than arrows between $A$

\begin{theorem}[2.11]
    If $0\to  G' \to  G \to  G'' \to  0$ is a short exact sequence of $R$-modules, then there is a long exact sequence
    \[
        \cdots \to \Ext_R^{n}(A, G) \to  \Ext_R^{n}(A, G'') \to  \Ext_R^{n+1}(A, G') \to  \cdots
    .\] 
\end{theorem}
Here arrows in same direction as arrows between $G$.

\begin{proof}
    \[
        0 \to  G' \to  G \to  G'' \to  0
    \]
    We apply $\Hom_{R}(P_n, -)$ to this ses.
    Then we have
    \[
     0 \to  \Hom_{R}(P_n, G') \to  \Hom_{R}(P_n, G) \to  \Hom_{R}(P_n, G'') \to  0
    ,\] 
    and because $P_n$ is projective, this functor is exact, so this sequence is exact.
    We can do this at any level, e.g.\ $P_{n+1}$.
    Now we can find maps between
    $\Hom_{R}(P_n, G') \to \Hom_{R}(P_{n+1}, G')$
    $\Hom_{R}(P_n, G) \to \Hom_{R}(P_{n+1}, G)$
    $\Hom_{R}(P_n, G'') \to \Hom_{R}(P_{n+1}, G'')$.
    Then we have cochaincomplexes vertaal and this then becomes a short exact sequence in cochain complexes and this gives rise to long exact sequence in homology.

\end{proof}

\begin{theorem}[2.12]
    Let $P$ be an $R$-module.
    Then $P$ is projective iff $\Ext_R^{n}(P, M) = 0$ for all $n \ge 1$ and all $R$-modules $M$.
\end{theorem}
\begin{proof}
    $\implies$. When $P$ is projective, then  the following is a projective resolution:
    \[
    \cdots \to  0 \to  0 \to  P \xrightarrow{\epsilon = \operatorname{Id}}   P \to  0
    .\] 
    So applying Hom to deleted resolution:
    \[
    0 \to  \Hom_{R}(P, M) \to  \Hom_{R}(0, M) \to  \cdots
    .\] 
    Now we have to compute the cohomology, so zero if $n \ge 1$.

    $\impliedby$. Other direction.
    Assume that $\Ext_R^{n}(P, M) = 0$ for all $n \ge 1$ and for all $M$.
    Let
     \[
    0 \to  A \to  B \to  C \to  0
    \] 
    be a ses of modules.
    Then $P$ is a projective module iff $ \Hom_{R}(P, -)$ applied to this short exact sequence is again a short exact sequence.

    We do get a long exact sequence of Ext functors.

    \[
        0 \to  \Hom_{R}(P, A) \to \Hom_{R}(P, B) \to \Hom_{R}(P, C) \to  \Ext_R^{1} (P, A) \to  \cdots
    .\] 
    So we have, assume exts are zero:
    \[
        0 \to  \Hom_{R}(P, A) \to \Hom_{R}(P, B) \to \Hom_{R}(P, C) \to  0
    ,\] 
    so ses so $P$ is projective module.
\end{proof}

Where does name come from?

\begin{definition}
    An extension of an $R$-module $C$ by $A$ is a ses
    \[
    0 \to A \to  B \to  C \to  0
    .\] 
\end{definition}
If we know something about Ext then we know something about the extensions.
For example, there always exists the trivial extension:
\[
0 \to  A \to A \oplus C \to  C \to  0 
.\] 
\begin{definition}
    Two extensions of $C$ by $A$ are equivalent iff
    \[
        \begin{tikzcd}
            0 \arrow[r, ""] & A\arrow[d, "Id"] \arrow[r, ""] & B \arrow[d, "", dashed]\arrow[r, ""] & C \arrow[r, ""] \arrow[d, "Id"]& 0\\
            0 \arrow[r, ""] & A \arrow[r, ""] & B' \arrow[r, ""] & C \arrow[r, ""] & 0\\
        \end{tikzcd}
    \]
    the dashed arrow exists and commutes. It will automatically be an isomorphism.
\end{definition}
\begin{eg}
    For example, an extension of $C$ by $A$ is equivalent to  the trivial extension $0 \to  A \to  A \oplus C \to  C \to  0$ iff the extension splits.
\end{eg}

\begin{prop}[2.13]
    If $\Ext_R^{1}(C, A)$ is zero, then every extension of $C$ by $A$ splits!
\end{prop}
\begin{proof}
    Let $0 \to  A \to B \to  C \to  0$ be an extension.
    Then there is a long exact sequence of ext functors.
    We will fix $C$ and put  $A$, $B$, and $C$ at the second place.

    \[
        0 \to  \Hom_{R}(C, A) \to  \Hom_{R}(C, B) \to \Hom_{R}(C, C) \xrightarrow{p_{*}}  \Ext_R^{1}(C, A) = 0
    ,\] 
    so $p_*$ is surjective.
    So there exists a morphism  $C \to  B$ which is mapped to $\operatorname{Id}: C \to  C$.  Remember that $p_*(s) = p  \circ  s$. So indeed, $s$ is a section $C\to  B$.
    So the sequence splits.
\end{proof}

Even more is true: set of equivalence relations of extensions is in one to one correspondence with $\Ext_R^{1}(C, A)$.
