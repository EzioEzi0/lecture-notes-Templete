\lesson{1}{di 29 sep 2020 10:30}{Introduction}


Course consists of three parts:
\begin{enumerate}
    \item Representation Theory (easy)
    \item Nilpotent groups and polycyclic groups (Kimpe's groups)
    \item (Co)homology of groups
\end{enumerate}


Exam:
\begin{itemize}
    \item List of 10 problems (hand in before the exam) \hfill 8/20
    \item Written exam (3 hours). Open book. Three question, one from each part, each for 4 points.
        Questions comparable to problems, or explain proof of a theorem in detail. \hfill 12/20
\end{itemize}

Exercise sessions:
\begin{itemize}
    \item Given by Bramm Verjans
    \item Everyone allowed to attend? Let's see what happens \ldots Be on time!
\end{itemize}

\part{Representation Theory}

\nchapter{2}{Representations of finite groups}

\section{Linear representations}

In what follows, let $G$ be a finite group; $k$ or  $K$ will always be a field of characteristic $0$; $V$ is a finite dimensional vector space over $k$.

\begin{definition}
    A linear representation of $G$ (over $V$) is a morphism of groups 
    \[
        \rho: G \to  \GL_k(V)
    .\] 
\end{definition}

So each $\rho(g) \in \GL_k(V)$ is a linear map.
We denote
\[
    \rho(g) (v) = g\cdot v = {}^{g} v
.\] 

We say that \emph{$G$ acts on $V$ } and \emph{$V$ is a $G$-module}.


Often, we fix a basis $ v = \{v_1, \ldots, v_n\}$ of $V $ and then we denote the matrix of $\rho(g)$ w.r.t. $v$ as $[r(g)]_v = (r_{ij}(g))$.
We put images of the $i$th basis vector in the $i$th column, i.e.
\[
    \rho(g)(v_i) = \sum_{k=1}^{n} r_{ki} v_k
.\] 

\begin{eg}[Representations to small vector spaces]
    Let $\dim V = 1$
    Then if $\rho: G\to \GL_k(V)$ is a representation, then for all $g \in G$, there exists a $\lambda_g \in k^{*}$ such that\footnote{Here we use $k^{*}$ to denote invertible elements of $k$}
    \[
        \rho(g)(v) = \lambda_g v \qquad \forall  v \in V
    .\] 
    Moreover, $\rho(gh) = \rho(g) \rho(h)$.
    So we have that  $\lambda_{gh} = \lambda_g \lambda_h.$
    Therefore, $\rho$ corresponds to a morphism
    \[
    \lambda: G \to  k^{*}
    .\] 
    Moreover, $G$ is finite. Therefore, $\lambda_g^{|G|} = \lambda_e = 1$.
    %TODO So $\im \lambda \subset \{\text{order-of-$G$ roots of $1$ in $k^{*}$}\}$.
\end{eg}

\begin{eg}[Representations from small groups]
    The trivial group $G = 1$ is trivial.
    $G = \Z_2 = \left<1, \tau \right>$ with $\tau^2 = 1$.
    Let $\rho: G \to  \GL_k(V)$ be a representation.
    Define the following linear maps:
    \begin{align*}
        \epsilon_+&: V \to  V : v \mapsto  \frac{1}{2} (v + \tau\cdot v)\\
        \epsilon_-&: V \to  V : v \mapsto  \frac{1}{2} (v - \tau\cdot v)\\
    .\end{align*} 
    (We can divide by two because char = 0. This doesn't work for char = 2)

    Claim. Let $w \in  \im (\epsilon_+)$, which is a subspace of $V$.
    Then $\tau\cdot w$, with $w = \frac{1}{2}(v + \tau\cdot v)$, equals 
    \[
        \tau\cdot w = \frac{1}{2}(\tau v + \tau \tau v) = 
        \frac{1}{2}(\tau v + 1 v)  = w
    .\] 

    Let $w \in  \im (\epsilon_-)$, then
    \[
    \tau\cdot w = - w
    .\]

    So we have two subspaces, one on which $\tau$ acts on as the identitiy, and another on which $\tau$ is doing $\cdot -1$.

    Claim:
    \[
        V = \im (\epsilon_+)  \oplus \im(\epsilon_-)
    .\] 

    \begin{enumerate}[(a)]
        \item $\im(\epsilon_-) \cap \im(\epsilon_+) = 0$. Indeed, pick a vector in the intersection. Then apply $\tau$ and we get that $v = -v$, so  $v = 0$.
        \item  $\im(\epsilon_+) + \im(\epsilon_-) = V$. Easy:
             \[
                 v = \frac{1}{2} (v + \tau v) + \frac{1}{2}(v - \tau v)
            .\] 
    \end{enumerate}

    So a representation of $\Z_2$ is a reflection of a subspace. (This subspace could also be empty, or the whole space)
\end{eg}


\begin{definition}[$G$-subspace]
    Let $V$ be a $G$-module.
    A $G$-subspace, $G$-submodule, $G$-subrepresentation  is a subspace $W$ of $V$ such that $G\cdot W \subset W$, and hence $G \cdot W = W$.\footnote{Map is invertible and hence preservers dimension.}
\end{definition}

\begin{eg}
    The subspaces $\im(\epsilon_+)$ and  $\im(\epsilon_-)$ are $G$ subspaces of $V$. (where $G = \left<1, \tau \right>$).
    Also $0$ and $V$ are $G$-subspaces.
\end{eg}

If $W$ is a $G$-subspace of $V$.
Then there is an induced (restricted) representation:
\[
    \rho|_W : G \to \GL_k(W) : g \mapsto (w \mapsto \rho(g)(w))
.\] 
There is also an induced representation on the quotient:
\[
    \overline{\rho}: G \to  \GL_k(V / W) : g \mapsto (v + W \mapsto  \rho(g)(v) + W)
.\]


\begin{remark}
    Let $ w_1, w_2, \ldots, w_k$ be a basis of $W$ and extend this to a basis $v = \{w_1, w_2, \ldots, w_k, v_{k+1}, \ldots, v_n\}$ of $V$.
    Then
    \[
        [\rho(g)]_v = 
        \begin{pmatrix}
            [\rho|_W(g)]_w & * \\
            0 & [\overline{\rho}(g)]_{\overline{v}}
        \end{pmatrix}
    ,\] 
    where $\overline{v} = \{\overline{v_k}, \ldots, \overline{v_n}\}$.
\end{remark}

\begin{remark}
    We will show that every group has good representations.
    The number of conjugacy classes determines the number of irreducible representations!
\end{remark}

\section{$G$-homomorphisms and irreducible representations}


\begin{definition}[$G$-homomorphisms]
    Let $\rho: G \to  \GL_k(V)$ and $\sigma: G \to  \GL_k(W)$ be two representations.
    A \emph{$G$-map (or $G$-equivariant map, $G$-homomorphism)} from $V$ to $W$ is a linear map $f: V \to  W$ such that  for all $g \in G$ and $v \in V$:
    \[
        f(g\cdot v) = g\cdot f(v)
    .\] 
    With more precise notation:
    \[
        f(\rho(g)(v)) = \sigma(g)(f(v))
    .\] 

    The map $f$ is a \emph{$G$-equivalence} ($G$ isomorphism) if $f$ is an isomorphism of vector spaces and a  $G$-map

\end{definition}

In other words, this definition says that the following diagram is commutative:
\[
    \begin{tikzcd}
        V \arrow[r, "f"] \arrow[d, "\rho(g)"]& W \arrow[d, "\sigma(g)"]\\
        V \arrow[r, "f"] & W
    \end{tikzcd}
\]


What does this imply for matrices?
Choose basis $\mathbf{v}$ of $V$ and  $\mathbf{w}$ of  $W$.

% \[ TODO
%     \begin{tikzcd}
%         V \arrow[r, "f = A"] \arrow[d, "B(g) = [\rho(g)]_v"]& W \arrow[d, "C(g) = [\sigma(g)]_w"]\\
%         V \arrow[r, "f = A"] & W
%     \end{tikzcd}
% \]

Note that $A$ is not necessarily square.
Then we get:
\[
    A \cdot  B(g) = C(g) \cdot A
.\] 

Suppose that $V = W$ and  $\rho = \sigma$ and  $\mathbf{v} = \mathbf{w}$.
Then we have
\[
    A \cdot B(g) = B(g)\cdot  A
.\] 

So $G$ maps are represented by matrices that commute with all $B(g)$'s.
We always have the $0$ map,  identity map and scalar multiplication.
By looking at these matrices, we can also solve for $A$.


\begin{definition}
    Let $\rho: G \to  \GL_k(V)$ be a representation.
    Then $\rho$ is irreducible iff the only $G$-subspaces of $V$ are $0$ and $V$.
\end{definition}
\begin{eg}
    One dimensional representations are always irreducible.
\end{eg}

\begin{prop}[2.7]
    Let $\rho: G > \GL_k(V)$ and  $\sigma: G \to  \GL_{k}(W)$ be two representations and $f: V \to  W$ a $G$-map.
    Then $\Ker f$ is a $G$-subspace of  $V$ and $\im(f)$ is a $G$-subspace of $W$.
\end{prop}
\begin{proof}
    Note that $\Ker(f)$ and  $\im(f)$ are subspaces.
    Now we check that they are equivariant under $G$-action.

    \begin{enumerate}[(a)]
        \item Let $v \in \Ker f$.
            Then $ f(\rho(g) v) = \sigma(g) f(v) = \sigma(g)(0) = 0 .$ 
            So $\rho(g)(v) \in \Ker f$
        \item Let $w  = f(v) \in \im f$.
            Then $ g\cdot w = g \cdot  f(v) = f (g\cdot v) \in \im f
            .$ 
    \end{enumerate}
\end{proof}


One of the most important results in representation theory:

% If you have 
\begin{theorem}[2.8, Schur's lemma]
    Let $\rho: G \to  \GL_k(V)$, $\sigma: G \to  \GL_k(W)$ be two \emph{irreducible} representation and $f: V \to  W$ a $G$-map. Then
    \begin{enumerate}[(a)]
        \item Either $f \equiv 0$, or $f$ is a $G$-isomorphism. (This implies that $\dim V  = \dim W$ if  $f \not\equiv 0$.)
        \item If $k = \C$ and $V = W$ and  $\rho = \sigma$.
            Then there exists $\lambda \in \C$ such that $f(v) = \lambda v$.
            (So the only $G$-maps are $0$, identity and multiplication by scalar)
    \end{enumerate}
\end{theorem}

\begin{proof}
    \begin{enumerate}[(a)]
        \item $\Ker(f)$ is a  $G$-subspace of $V$.
            \begin{itemize}
                \item Because representation is irreducible, we have that $\Ker(f) = 0$ or  $\Ker (f) = V$.
                    In the last case,  $f = 0$.
                    In the first case,  we know that $f$ is injective and not identically $0$.
                \item Because the representation is irreducible we have that $\im(f) = 0$ or  $\im(f) = W$. The first case has been handled.
                    In the other case, $f$ is onto.
            \end{itemize}
            Conclusion: $f$ is bijective. So  $f$ is a $G$-isomorphism.
        \item Take $\lambda$ an eigenvalue of the linear map $f$ and $v_0$ a non-zero eigenvector corresponding to $\lambda$. (Here we are using the fact that $\C$ is algebraically closed.) 

            Let  $f_\lambda: V \to  V: v \mapsto  f(v) - \lambda v$.
            Note that $f_\lambda$ is a linear map.

            Now we want to show that  $f$ commutes with $G$-action:
            \[
                f_\lambda ( g\cdot v ) = f(g\cdot v) - \lambda (g\cdot v)
                = g \cdot f(v)  - g\cdot (\lambda v) = g\cdot (f(v) - \lambda v) = g\cdot f_\lambda(v)
            .\] 
            Conclusion: $f_\lambda$ is a  $G$-linear map.

            We have proven in point $(a)$ that $f_\lambda$ is either a $0$ or an isomorphism.
            Note that $f_\lambda(v_0)  = f(v_0) - \lambda v_0 = 0$
            So we have a non-zero vector mapped to a vector that is zero.
            So $f_\lambda$ is not an isomorphism. Conclusion:  $f_\lambda = 0$
            or $f(v) = \lambda v$ for all  $v$.
    \end{enumerate}
\end{proof}

% TODO: macro repr
